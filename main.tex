\documentclass[11pt, a4paper,]{scrartcl}
%\setkomavar{subject}{Title of essay?} % Comment out if you don't need a subject line
\usepackage[utf8]{inputenc}
\usepackage{charter}
\usepackage{ragged2e}
\usepackage[margin=2cm]{geometry}
%-----------------------------------------------------------------
\setlength{\parindent}{0em}
\setlength{\parskip}{1em}
%\linespread{1.5}

%-----------------------------------------------------------------
\begin{document}
\vspace{-2em}
\begin{flushleft}{\footnotesize{}
PICT711 - Research Essay\\
Aaron Hammond - 43691455\\
Word Count:
}
\end{flushleft}
%------------------------------------------------------------------

\hrule
%-------------------------------------------------------------------
\begin{center}
\huge Collectivizing Harm
\end{center}
%-------------------------------------------------------------------

Choose a case of corporate or state crime and use it to illustrate
and discuss the theoretical and empirical issues discussed in class\par

The collectivization of farms within the Soviet Union was a policy spearheaded by the General Secretary of the Communist Party of the Soviet Union, Joseph Stalin. During the years from 1928 to 1940, the Soviet Union looked to transform the agricultural sector. Stalin's motivation for the collective farm initiative were twofold. Primarily, the policy would dismantle the capitalist model of private farming purported by the Kulak class. The Kulaks were a class perceived somewhat as 'prosperous peasants' who maintained the land and sold its yield. As a result, the Kulak class was perceived as a threat to socialism as they would often hold onto their grain reserves for a more prosperous time in the market. Stalin's policy of collectivization would disenfranchise the Kulaks and place their land in state control. Secondly, Stalin believed that collectivized farms could pave the way for industrialization and provide a greater economical and agricultural yield for the state and its people. The collective-farm policy was anticipated as a force of mobilization to combat grain shortages. The policy saw some success in the sheer quantity of grain that was produced however the restrictions on workers from withholding grain thoroughly undermined their efforts. This, combined with high work-quotas saw that peasants then received less for their labour than before. Ultimately the reality of output of the farms did not meet the projections. As a result Stalin enacted the \textit{Ural-Siberian method} to rectify the poor results. What began as a tool of social coercion transformed into a process of genocide. It provided the state with the power to dispossess, deport and execute \cite{Hughes, 1994} individuals perceived to be sabotaging the collective farm policy. A set of punishments were set in place that would incur heavy fines for not meeting grain quotas, followed by the acquisition of their property by the state. The final punishment was exile and execution, a fate that was endured by over 10 million individuals. While this policy was enacted legally, a harm-based approach to the subject suggests its a state-crime of genocide.
\par
The Ural-Siberian method was a form of state-crime that resulted in the deaths of over 10 million people. Stalin's introduction of the Ural-Siberian method through state apparatus ensured that the states actions were not illegal, however through the lens of harm-based definitions it is clear that the policy is a form of state-enacted genocide. 


Define what the criteria is for state crime and how the collectivization of farms fits (harm definition) \par

Analyze a text from Stalin's address the justifies his actions and his policy using the various forms of  techniques of neutralisation. Dizzy with success as a response to farmers resisting collectivization. Liquidate\par

Using a form of hard control (violence)? \par








% \cite{Robbins2017}.

%--------------------------------------------------------------------

\newpage
\bibliographystyle{apalike}
\bibliography{ref.bib}

%--------------------------------------------------------------------
%\newpage
%\begin{thebibliography}{}
%\bibitem{article}
%Gat, O. (2018). Estonia goes digital: Residents of the tiny Baltic nation are going all %in on techno-governance. World Policy Journal, 35(1), 108–113. %https://doi.org/10.1215/07402775-6894885
%\bibitem{book}
%Robbins, J. (2017). Two Lenins. In HAU Books.
%\end{thebibliography}
%---------------------------------------------------------------------
\end{document}